%%%%%%%%%%%%%%%%%%%%%%%%%%%%%%%%%%%%%%%%%
%%%%%%%%%% Content starts here %%%%%%%%%%
%%%%%%%%%%%%%%%%%%%%%%%%%%%%%%%%%%%%%%%%%


\begin{frame}[fragile]{Remainding From Lecutres}
% Define block styles
\tikzstyle{decision} = [diamond, draw, fill=blue!20, 
    text width=4.5em, text badly centered, node distance=3cm, inner sep=0pt]
\tikzstyle{block} = [rectangle, draw, fill=blue!20, 
    text width=5em, text centered, rounded corners, minimum height=4em]
\tikzstyle{line} = [draw, -latex']
\tikzstyle{cloud} = [draw, ellipse,fill=red!20, node distance=3cm,
    minimum height=2em]
    
\begin{center}
\begin{tikzpicture}[node distance = 2cm, auto]
    % Place nodes
	\node [decision] (decide) {Condition};
    \node [block, left of=decide, below of=decide] (true) {Expressions
    for true condition}; \node [block, right of=decide,
    below of=decide] (false) {Expressions for false condition};
    % Draw edges
    \path [line] (decide) -| node [near start] {true} (true);
    \path [line] (decide) -| node [near start] {false} (false);
\end{tikzpicture}
\begin{lstlisting}
if(condition) {
    expressions_for_true_condtion;
} else {
    expressions_for_false_condition;
}
\end{lstlisting}
\end{center}
\end{frame}

\begin{frame}[fragile]{\texttt{if} usage}
	\begin{exampleblock}{Example 1}
	\begin{lstlisting}
	#include <stdio.h> 
	int main() { 
	    int i;
	    printf("Enter an integer\n");
	    scanf("%d", &i);
	    if(i > 0) 
	        printf("The number is positive\n");
	    if(i < 0)
	        printf("The number is negative\n"); 
	    if(i == 0)
	        printf("The number is zero\n"); 
	    return 0; 
	}
	\end{lstlisting}
	\end{exampleblock}
\end{frame}

\begin{frame}[fragile]{By using \texttt{if-else}}

	\begin{exampleblock}{Example 2}
	\begin{lstlisting}
	#include <stdio.h>
	int main() {
	    int i;
	    printf("Enter an integer\n");
	    scanf("%d", &i); 
	    if(i > 0)
	        printf("The number is positive\n");
	    else if(i < 0)
	        printf("The number is negative\n"); 
	    else
	        printf("The number is zero\n"); 
	    return 0; 
	}
	\end{lstlisting}
	\end{exampleblock}

\end{frame}

\begin{frame}[fragile]{Simple examples}
What will be the output?
\begin{exampleblock}{Example 3}
	\begin{lstlisting}
#include <stdio.h>
int main() {
    int m = 5, n = 10;
    if(m > n)
        ++m;
    ++n;
    printf("m = %d, n = %d\n", m, n);
    return 0;
} 
\end{lstlisting}
\end{exampleblock}
\pause
\vfill
\texttt{m = 5, n = 11}
\end{frame}


\begin{frame}[fragile]{Problem 1}
Write a program that prints out the maximum from two numbers read from
standard input.
\pause 
	\begin{exampleblock}{Solution}
	\lstinputlisting{src/av3/p1.c}
	\end{exampleblock}

\end{frame}

\begin{frame}[fragile]{Problem 2}
Write a program that checks if given year read from SI is leap or not and prints
out a appropriate message.\\
Example: \texttt{1976, 2000, 2004, 2008, 2012...}
\pause
	\begin{exampleblock}{Solution}
	\lstinputlisting{src/av3/p2.c}
	\end{exampleblock}
\end{frame}


\begin{frame}[fragile]{Problem 3}
\begin{scriptsize}
The coordinates of a point are read from SI. Write a program that will print out
the quadrant or the axis where the point belongs. If the point lays on the
origin, print out a appropriate message.
\end{scriptsize}
\pause
	\begin{exampleblock}{Solution part 1}
	\lstinputlisting[lastline=11]{src/av3/p3.c}
	\end{exampleblock}
\end{frame}



\begin{frame}[fragile]{Problem 3}{Solution}
	\begin{exampleblock}{Solution part 2}
	\lstinputlisting[firstline=12]{src/av3/p3.c}
	\end{exampleblock}
\end{frame}



\begin{frame}[fragile]{Problem 4}
Write a program that will generate and print the grade according to the
following table:
\begin{center}
\begin{tabular}{|c|c|}
\hline \textbf{Points} & \textbf{Grade} \\ 
\hline 0 - 50 & 5 \\ 
\hline 51 - 60 & 6 \\ 
\hline 61 - 70 & 7 \\ 
\hline 71 - 80 & 8 \\ 
\hline 81 - 90 & 9 \\ 
\hline 91 - 100 & 10 \\ 
\hline
\end{tabular} 
\end{center}
\end{frame}

\begin{frame}[fragile]{Problem 4}{Solution}
    \begin{exampleblock}{Solution}
    \lstinputlisting[lastline=11]{src/av3/p4.c}
	\end{exampleblock}
\end{frame}

\begin{frame}[fragile]{Problem 5}
Change the previous program, so the sign of the number should be printed (+/-)
depending on the last digit of the points number:
\begin{center}
\begin{tabular}{|c|c|}
\hline \textbf{last digit} & \textbf{print} \\ 
\hline 1 - 3 & - \\ 
\hline 4 - 7 & <empty space> \\ 
\hline 8 - 0 & + \\ 
\hline 
\end{tabular} 
\end{center}
Example: \texttt{81 = 9-, 94 = 10, 68 = 7+}.\\ 
For grade 5 doesn't add + or –, and for grade 10 should not add +.
\end{frame}


\begin{frame}[fragile]{Problem 5}{Solution}
	\begin{exampleblock}{Solution}
	\lstinputlisting[firstline=13]{src/av3/p5.c}
	\end{exampleblock}
\end{frame}


\begin{frame}[fragile]{Problem 6}
Write a program for simple calculator. The program reads two numbers and
operator in \texttt{format}:
\begin{center}
\texttt{num1 operator num2}
\end{center}
After the operation, depending on the operator, the result shoud be printed in
format:
\begin{center}
\texttt{num1 operator num2 = result}
\end{center}
\end{frame}

\begin{frame}[fragile]{Problem 6}{Solution}
	\begin{exampleblock}{Solution}
	\lstinputlisting{src/av3/p6.c}
	\end{exampleblock}
\end{frame}


\begin{frame}[fragile]{Problem 7}
Read from standard input three numbers in arbitrary order. The numbers 
are lenghts of triangle sides. Write a program that will check if triangle can
be constructed from given lengths, if so, than should check if the triangle is
right triangle and compute its area. On contrary, apropriate messages should be
printed.
\end{frame}


\begin{frame}[fragile]{Problem 7}{Solution}
	\begin{exampleblock}{Solution}
	\lstinputlisting{src/av3/p7.c}
	\end{exampleblock}
\end{frame}

