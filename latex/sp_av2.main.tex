%%%%%%%%%%%%%%%%%%%%%%%%%%%%%%%%%%%%%%%%%
%%%%%%%%%% Content starts here %%%%%%%%%%
%%%%%%%%%%%%%%%%%%%%%%%%%%%%%%%%%%%%%%%%%


\begin{frame}{Потсетување од предавања}
\begin{itemize}
	\item Оператори
	\begin{itemize}
		\item Аритметички
		\item Релациски
		\item Логички
	\end{itemize}
	\item Печатење на стандарден излез
	\texttt{printf()}
	\item Читање од стандарден влез
	\texttt{scanf()}
\end{itemize}
\end{frame}

\begin{frame}[fragile]{Читање од стандарден влез во C}{Функцијата
\texttt{scanf}}
	\begin{verbatim}
	int scanf(Контролна_низа_од_знаци, arg1, arg2, ..., argn)	
	\end{verbatim}	
	\begin{itemize}
	\item Контролната низа од знаци е всушност низа од знаци којa ја содржи потребната информација за форматирање	
	\item arg1, arg2, ..., argn се аргументите кои ги претставуваат индивидуалните податоци
	\end{itemize}	 
\end{frame}

\begin{frame}[fragile]{Употреба на \texttt{scanf}}

	\begin{exampleblock}{Пример 1}
	\begin{lstlisting}
	#include <stdio.h>
	int main() {
	    char del;
	    int delbroj;
	    float cena;
	    scanf("%c%d%f", &del, &delbroj, &cena);
	    return 0;
	}
	\end{lstlisting}
	\end{exampleblock}

\end{frame}

\begin{frame}[fragile]{Задачa 1}
Да се напише програма за пресметување и печатење на плоштината и периметарот на круг. 
Радиусот на кругот се чита од тастатура како децимален број.
\begin{exampleblock}{Решение}
\lstinputlisting{src/av2/p1.c}
\end{exampleblock}
\end{frame}


\begin{frame}[fragile]{Задача 2}
Да се напише програма која од стандарден влез ќе прочита два цели броја и ќе ја
испечати нивната сума, разлика, производ и остатокот при делењето.
	\begin{exampleblock}{Решение}
		\lstinputlisting{src/av2/p2.c}
	\end{exampleblock}
\end{frame}

\begin{frame}[fragile]{Задача 3}
Да се напише програма која чита големa буквa од стандарден влез и ја печати
истaтa како малa буквa.\\ Помош: Секој знак се претставува со ASCII број.\\
Пр. \texttt{'А' = 65, 'а' = 97}
	\begin{exampleblock}{Решение}
		\lstinputlisting{src/av2/p3.c}
	\end{exampleblock}
\end{frame}

\begin{frame}[fragile]{Задача 4}
\begin{scriptsize}
Да се напише програма која чита знак од стандарден влез и во зависнот од тоа
дали е мала или голема буква печати 1 или 0, соодветно.\\ Помош: Користете
логички и релациони оператори за тестирање на ASCII вредноста на знакот.\\ 
\textbf{Бонус:} Направете проверка дали знакот е број	
\end{scriptsize}
\begin{exampleblock}{Решение}
	\lstinputlisting{src/av2/p4.c}		
\end{exampleblock}
\end{frame}

\begin{frame}[fragile]{Задача 5}
\begin{scriptsize}
Да се напише програма која ќе чита два цели броеви (x, y) од стандарден влез и
на стандарден излез ќе го испечати резултатот (z) од следниот израз\\ \texttt{z
= x++ + ---y + (x<y)}\\ Каква вредност ќе има z за x=1, y=2?
\end{scriptsize}
	\begin{exampleblock}{Решение}
		\lstinputlisting{src/av2/p5.c}
	\end{exampleblock}
\end{frame}

\begin{frame}[fragile]{Задача 6}
Нека е дадено: \texttt{r = (x < y || y < z++)}\\
	Каква вредност ќе има r за x=1, y=2, z=3?\\
	Каква вредност ќе има z?
	\begin{exampleblock}{Решение}
	\texttt{r = 1\\z = 3}
	\end{exampleblock}
Нека е дадено: \texttt{r = (x > y \&\& y < z++)}\\
	Каква вредност ќе има r за x=1, y=2, z=3?\\
	Каква вредност ќе има z?
	\begin{exampleblock}{Решение}
	\texttt{r = 0\\z = 3}
	\end{exampleblock}
\end{frame}

\begin{frame}[fragile]{Задача 7}
Нека е дадено:
\begin{lstlisting}
	int x, y;
	y = scanf("%d", &x);
\end{lstlisting}
Каква вредност ќе има y за x=5?
	\begin{exampleblock}{Решение}
	\texttt{y = 1}
	\end{exampleblock}
Нека е дадено:
\begin{lstlisting}
	int x, y, z;
	z = scanf("%d%d", &x, &y);
\end{lstlisting}
	Каква вредност ќе има z за x=5, y=6?
	\begin{exampleblock}{Решение}
	\texttt{z = 2}
	\end{exampleblock}
\end{frame}


\begin{frame}[fragile]{Задача 8}
Да се напише програма каде од стандарден влез ќе се прочита цена на производ, а
потоа ќе ја испечати неговата цена со пресметан ДДВ.\\ Помош: ДДВ е 18\% од почетната цена
	\begin{exampleblock}{Решение}
		\lstinputlisting{src/av2/p8.c}
	\end{exampleblock}
\end{frame}


\begin{frame}[fragile]{Задача 9}
Да се напише програма каде од стандарден влез се чита цена на производ, број на
рати на кои се исплаќа и камата (каматата е број изразен во проценти од 0 до 100). 
Програмата треба да го испечати износот на ратата и вкупната сума што ќе се исплати за производот.\\
Помош: Пресметајте ја целата сума, па потоа ратата.
\end{frame}

\begin{frame}[fragile]{Задача 9}{Решение}
	\lstinputlisting{src/av2/p9.c}
\end{frame}


\begin{frame}[fragile]{Задача 10}
Да се напише програма каде од стандарден влез се чита еден трицифрен цел број.
Програмата ќе ја испечати најзначајната и најмалку значајната цифра од бројот\\ 
Пример: Ако се внесе следниот бројот 795, програмата ќе испечати:\\
\texttt{	Najznacajna cifra e 7, a najmalku znacajna e 5.}\\
Помош: Искористете целобројно делење и остаток од делење.
\end{frame}

\begin{frame}[fragile]{Задача 10}{Решение}
\lstinputlisting{src/av2/p10.c}
\end{frame}


\begin{frame}[fragile]{Задача 11}
Да се напише програма каде од стандарден влез се вчитува датумот на раѓање во
формат (\texttt{ddmmgggg}). Програмата стандарден излез треба да го испечати
денот и месецот на раѓање.\\ Пример: Ако се внесе следниот број
\texttt{18091992}, програмата ќе испечати: \texttt{18.09}\\ Помош: Искористете целобројно делење и остаток од делење.
\end{frame}

\begin{frame}[fragile]{Задача 11}{Решение}
	\begin{exampleblock}{Решение}
	\lstinputlisting{src/av2/p11.c}	
	\end{exampleblock}
	Бонус: Задачата може да се реши и со употреба на \texttt{scanf("\%2d\%2d", \&den, \&mesec)}
\end{frame}
