* Коja од наведените можности не е структура за повторување во C:

for
loop until
while
do while

* Што ќе се отпечати на екран по извршување на следниов програмски код:

int x=23;
switch(x) {
  case 1: printf( "1" ); break;
  case 23: printf( "2" ); break;
  case 123: printf( "3" ); break;
}

123
1
23
2

* Која од следниве декларации не е валиден функциски прототип:

int funk(char,char);
double funkcija;
void funk();
float funkcija();

* Ако е даден функцискиот прототип: float ab(float,float,int) за една
рекурзивната функција ab, која од опциите претставува потенцијален валиден повик на рекурзивната функција:

    primer = ab(а, а, k - 1) * (k - 1) / k + ab(а - 1, b, k - 2) * 1 / n;
    primer = ab(x0, x1) * (k - 1) / k + ab(a, b) * 1 / n;
    primer = ab(0, 2.5, 3.0) * (k - 1) / k + ab(5.9, b) * 1 / n;
    primer = ab(x0, x1) + ab(3, b) * 1 / n;

* Што ќе се прикаже на екран како резултат од следниов програмски сегмент:      
    char *t = "prezime";
    t += 4;
    printf("%s", t); 

ништо, компајлерот ќе јави грешка
ime
prezime
qrezime


* Што ќе се испечати по извршување на следниот програмски код:
    int a = 3, *b, *c;
    b = &a;
    c = b;
    *c = 5;
    printf("%d\n", *b);

1
3
5
не може да се определи

* Која од следниве дефиниции на функции е комплетна:

int funk(int a) {return a*а;}        
void funk(char) { printf( "Zdravo");
void funk(y) { printf( "Zdravo"); } 
int funk();

* Со која од следниве наредби се читаат елементите од еднодимензионално поле со
име a од 5 целобројни елементи:

scanf(``%d'',a);               
for(i=0;i<5;i++) scanf(``%d",&a[i]);
for(i=0;i<5;i++) scanf(“%d”,&a[i]);          
for(i=1;i<=5;i++) scanf(“%d”,a[i]);               

* Со која од следниве наредби се пристапува до последниот елемент од полето
primer ако истото е деклaрирано на следниов начин: 

char primer[25];

primer(24);
primer[25];
*(primer + 24);                         
primer(25);

* Со која од следниве наредби се декларира целобројно дводимензионално поле  со
max 30 елементи:

int pole[3*10];
int pole[5][6];
int pole[30];
int pole[6,5];


* Што ќе се испечати по извршувањето на следниот програмски код: 

int j; char c; 
for(j=10,c='b';j;j-=2,c++);
 printf("%c\n",c);

кодот е невалиден
f        
g
l

* Што ќе се испечати по извршувањето на следниот програмски код: 

int x=3, y=1;
printf(“%d %d”,--x, y++);

2 1
2 2
3 1
3 2

* Кој е валиден програмски код во програмскиот јазик С кој проверува дали 10 < x
< 100?

if(10<x<=100)
if((100>x)&&(x>10))          
if((100>=x)&&(x>10)) 
while((100>=x)&&(x>10))

* Кој од следниве изрази e валиден во програмскиот јазик C

for(x>0;x--) 
for(x=1)
while(10*10)
ниту еден

* Што ќе се прикаже на екран по извршување на следниот програмски сегмент: 

int x = 5;
if(-1<=x<1) printf("1");
else printf("0");

0
1
кодот не е валиден
2

* Што ќе се отпечати на екран по извршување на следниов програмски код: 

#include<stdio.h>
int main ( )
{
    int a = 1; 
        for (printf ("100") ;printf ("") ;printf ("%d", a++)) {printf("100");break;}
}

100100
1001001
1001002
100

* Што ќе се испечати по извршување на следниот програмски код:
int a = 0; 
printf(" %d ", printf(" %d", printf("%d", a)));

2 1 0
0 1 2 3
0 1 2
кодот е невалиден


* Што ќе се испечати по извршување на следниот програмски код:

int a = 1; 
switch(a)
{
             case 1:printf("3");
             case 2:printf("2");
             case 3:printf("1");
             default:printf("default");
}

3
321
321 default      
кодот е невалиден

* Што ќе се испечати по извршување на следниот програмски код:

int x = 4;
int y = 3/2 + 1/x + 1/2;
printf (“y = %03d\n”, y);

y = 1
y = 2.25
y = 1
y = 001

* Што ќе се прикаже на екран по извршување на следниов програмски сегмент: 

int x, y, temp, br =5;
x = 3; y = 5;
    do {
            temp = x; x = y; y = temp;
        br--;       
        }
    while (br);
    printf("%d %d %d\n",x,y,br);

3 5 0
5 3 0
5 3 1    
                             
* Што ќе се отпечати на екран по извршување на следниов програмски код:
char x='b';
switch(x)
{
  case 'a': printf( "Prva" );
  case 'b': printf( "Vtora" );
  case 'c': printf( "Treta Bukva" );
}

Treta bukva
Prva
VtoraTreta bukva
Vtora   

* Кое од наведените можности не е структура за повторување во C:

for
while
do until
do while

* Која од следниве декларации не е валиден функциски прототип:

int funk(char, char);
void funk();
double funk;
float x();

* Која е вредноста на x по извршување на кодот: 

int x; for(x=10; x>=0; x--) {} 

0
1
-1
9

* Која од следниве наредби не дава 2.5 како резултат ако променливите се
декларирани како int a=5, b=2;

a/(float)b;
float(a)/(float)b;
(float)(a/b);
(float)a/b;

* Која од следниве дефиниции на функции е комплетна:

void funk(char) { printf( "Zdravo");
void funk(y) { printf( "Zdravo"); } 
int funk(int a) {return a++;}        
int funk();

* Која од понудените можности е валиден повик на функцијата со име funk (под претпоставка дека таа постои и нема влезни параметри):

A) funk;                    Б) int funk();
В) funk x, y;               Г) funk();

8. Со која од следниве наредби се декларира поле со 5 целобројни елементи:

А) pole{5};             Б) int pole[5];
В) int pole[4];             Г) array int[5];

9. Со која од следниве наредби се пристапува до последниот елемент од полето spTest ако истото е деклирарано на следниов начин: int spTest[5];

A) spTest(5);               Б) spTest[4];
В) spTest[5];               Г) spTest;

10. Со која од следниве наредби се декларира дводимензионално поле (матрица) со димензија 50x50 (50 редици и колони):

A) int pole[50*50];         Б) int pole[50][50];
В) int pole[50,50];         Г) int pole[50];

1. Што ќе се отпечати на екран по извршување на следниов програмски код:
int x=2;
switch(x)
{
  case 1: printf( "1" ); break;
  case 2: printf( "2" ); break;
  case 3: printf( "3" ); break;
}

А) 3                    Б) 1
В) 23                   Г) 2   

2. Кое од наведените можности не е структура за повторување во C:

А) for                  Б) while
В) loop until               Г) do while

3. Која од следниве декларации не е валиден функциски прототип:

A) int funk(char,char);     Б) double funk;
В) void funk();             Г) float x();

4. Која од следниве наредби не дава 3.5 како резултат ако променливите се декларирани како int a=7, b=2;

A) a/(float)b;                  Б) float(a)/(float)b;
В) (float)a/b;                  Г) (float)(a/b);

5. Која е вредноста на y по извршување на кодот: 

int y; for(y=0; y<=10; y++) {} 

А) 9                        Б) 10
В) 11                       Г) 0



6. Која од понудените можности е валиден повик на функцијата со име funk (под претпоставка дека таа постои и нема влезни параметри):

A) funk;                    Б) int funk();
В) funk x, y;               Г) funk();

7. Која од следниве дефиниции на функции е комплетна:

A) int funk(int a) {return a*а;}        
Б) void funk(char) { printf( "Zdravo");
В) void funk(y) { printf( "Zdravo"); } 
Г) int funk();

8. Со која од следниве наредби се декларира поле со 10 симболи:

А) pole{10};                    Б) array char[10];
В) char pole[10];               Г) char pole[10];

9. Со која од следниве наредби се пристапува до последниот елемент од полето spTest ако истото е деклaрирано на следниов начин: int spTest[5];

A) spTest(4);                   Б) spTest[5];
В) *(spTest + 4);               Г) spTest(5);

10. Со која од следниве наредби се декларира целобројно дводимензионално поле (матрица) со димензија 50x50 (50 редици и колони):

A) int pole[50*50];         Б) int pole[50][50];
В) int pole[50];            Г) int pole[50,50];

1. Што ќе се отпечати на екран по извршување на следниов програмски код:
x=1;
switch(x)
{
  case 0: printf( "Nula" );
  case 1: printf( "Eden" );
  case 2: printf( "Zdravo svetu" );
}
А) EdenZdravo svetu                 Б) Nula
В) Zdravo svetu                 Г) Eden   

2. Кое од наведените можности не е структура за повторување:
А) for                  Б) if…else 
В) while                    Г) do while

3. Кој од следниве функциски прототипови не е дозволен:
A) int funk(char,char);     Б) void funk();
В) double funk              Г) char x();

4. Која е вредноста на x по извршување на кодот: 
int x; for(x=10; x>0; x--) {} 

А)0                     Б) 1
В) -1                       Г)9

5. Која од следниве наредби е валидна cast операција:
A) a(char);                 Б) to(char, a);
В) (char)a;                 Г) char:a;

6. Која од следниве дефиниции на функции е комплетна:

А) void funk(char) { printf( "Zdravo");
Б) void funk(y) { printf( "Zdravo"); } 
В) int funk();      
Г) int funk(int a) {return a++;}

7. Која од понудените можности е валиден повик на функцијата funk (под претпоставка дека таа постои):

A) funk;                    Б) funk();
В) funk x, y;               Г) int funk();

8. Со која од следниве наредби се декларира поле со 5 целобројни елементи:

А) pole{5};                 Б) int pole[5];
В) int pole[4];             Г) array pole[5];

9. Со која од следниве наредби се пристапува до елемент од полето spTest ако истото е деклирарано на следниов начин: int spTest[5];

A) spTest[4];               Б) spTest(-1);
В) spTest[5];               Г) spTest;

10. Со која од следниве наредби се декларира дводимензионално поле (матрица) со 100 целобројни елементи:

A) int pole[100][100];          Б) int pole[50][50];
В) int pole[10][10];                Г) int pole[100];


1. Коja од наведените можности не е структура за повторување во C:

А) for                  Б) loop until
В) while                           Г) do while

2. Што ќе се отпечати на екран по извршување на следниов програмски код:
int x=23;

switch(x)
{
  case 1: printf( "1" ); break;
  case 23: printf( "2" ); break;
  case 123: printf( "3" ); break;
}

А) 123                  Б) 1
В) 23                   Г) 2   


3. Која од следниве декларации не е валиден функциски прототип:

A) int funk(char,char);     Б) double funkcija;
В) void funk();             Г) float funkcija();

 
4. Која од следниве наредби не дава 7.5 како резултат ако променливите се декларирани како int a=30, b=4;

A)(float)(a/b);                         Б) (float)a/(float)b;
В) (float)a/b;                  Г) a/(float)b;

5.  Ако е даден функцискиот прототип: float ab(float,float,int)за една рекурзивната функција ab, која од опциите претставува потенцијален валиден повик на рекурзивната функција:

           A) primer=ab(а,а,k-1)*(k-1)/k+ab(а-,b,k-2)*1/n;
           Б) primer=ab(x0,x1)*(k-1)/k+ab(a,b)*1/n;
В) primer=ab(0,2.5,3.0)*(k-1)/k+ab(5.9,b)*1/n; 
Г) primer=ab(x0,x1)+ab(3,b)*1/n;


6. Што ќе се прикаже на екран како резултат од следниов програмски сегмент :       char *t="prezime"; t+=4;     printf("%s",t);  ? 

А) ништо, компајлерот ќе јави грешка    Б) ime
В) prezime                  Г) qrezime

7. Која од понудените можности е валиден повик на функцијата со име funkcija (под претпоставка дека таа постои и има 2 влезни параметри):

A) funkcija;                    Б) int funkcija(x,y);
В) funkcija x, y;               Г) funkcija(x,y);

8. Која од следниве дефиниции на функции е комплетна:

A) int funk(int a) {return a*а;}        
Б) void funk(char) { printf( "Zdravo");
В) void funk(y) { printf( "Zdravo"); } 
Г) int funk();

9. Со која од следниве наредби се декларира поле со 12 симболи:

А) int pole[12];                    Б) array char[12];
В) pole{12};                                                Г) char pole[12];

10. Со која од следниве наредби се пристапува до последниот елемент од полето primer ако истото е деклaрирано на следниов начин: char primer[25];

A) primer(24);                  Б) primer[25];
В) *(primer + 24);              Г) primer(25);

11. Со која од следниве наредби се декларира целобројно дводимензионално поле  со max 30 елементи:

A) int pole[3*10];          Б) int pole[5][6];
В) int pole[30];            Г) int pole[6,5];

1. Коja од наведените можности е структура за повторување во C:

А) if…else                  Б) loop until
В) while...do               Г) do… while

2. Што ќе се отпечати на екран по извршување на следниов програмски код:
int x=23;

switch(x)
{
  case 1: printf( "1" ); break;
  case 2: printf( "23" ); break;
  case 23: printf( "123" ); break;
}

А) 123                  Б) 1
В) 23                   Г) 2   


3. Која од следниве декларации не е валиден функциски прототип:

A) int funk(char,char);     Б) double funkcija(int);
В) void funk;               Г) float funkcija();

 
4. Која од следниве наредби дава 4.0 како резултат ако променливите се декларирани како int a=13, b=3;

A)(float)(a/b);                         Б) (float)a/(float)b;
В) (float)a/b;                  Г) a/(float)b;

5.  Ако е даден функцискиот прототип: int f(float, int)за една рекурзивна функција, која од следните линии е грешна:

            A) printf("%d",f(a,2));                 Б) printf("%f",f(a,4));
В) p=f(a,2.0)+ f(b);                 Г) p=f(a-1,2)+ f(b,78); 



6. Што ќе се прикаже на екран како резултат од следниов програмски сегмент :       char *t="prezime"; t++;  printf("%s",t);  ? 

А) ништо, компајлерот ќе јави грешка    Б) prezime
В) ime                              Г) rezime

7. Која од понудените можности е валиден повик на функцијата со име funkcija (под претпоставка дека таа постои и има нема параметри):

A) funkcija();                  Б) int funkcija();
В) funkcija x;                          Г) funkcija;

8. Која од следниве дефиниции на функции е комплетна:

A) int funk(int a) { printf("%d",b);}       
Б) void funk(char k) { printf( "%c",k);}
В) void funk(y) { printf( "Zdravo"); } 
Г) int funk();

9. Со која од следниве наредби се декларира поле со 4 целобројни вредности:

А) pole{4};                        Б) array char[4];
В) int pole[4];                                 Г) char pole[4];

10. Со која од следниве наредби се пристапува до последниот елемент од полето primer ако истото е деклaрирано на следниов начин:  int primer[20];

A)  primer(20);                 Б) *(primer + 19);
В) primer(19);                          Г)  primer[21];

11. Со која од следниве наредби се декларира целобројно дводимензионално поле  со max 12 елементи:

A) int pole[2*6];           Б) int pole[2,6];
В) int pole[12];            Г) int pole[6][2];

1. Коja од наведените можности не е структура за повторување во C:

А) for                  Б) loop until
В) while                Г) do while

2. Што ќе се отпечати на екран по извршување на следниов програмски код:
int x=23;
switch(x)
{
  case 1: printf( "1" ); break;
  case 23: printf( "2" ); break;
  case 123: printf( "3" ); break;
}

А) 123                  Б) 1
В) 23                   Г) 2   


3. Која од следниве декларации не е валиден функциски прототип:

A) int funk(char,char);     Б) double funkcija;
В) void funk();         Г) float funkcija();

 
4. Ако е даден функцискиот прототип: float ab(float,float,int)за една рекурзивната функција ab, која од опциите претставува потенцијален валиден повик на рекурзивната функција:

           A) primer=ab(а,а,k-1)*(k-1)/k+ab(а-1,b,k-2)*1/n;
           Б) primer=ab(x0,x1)*(k-1)/k+ab(a,b)*1/n;
В) primer=ab(0,2.5,3.0)*(k-1)/k+ab(5.9,b)*1/n; 
Г) primer=ab(x0,x1)+ab(3,b)*1/n;


5. Што ќе се прикаже на екран како резултат од следниов програмски сегмент :       char *t="prezime"; t+=4;     printf("%s",t);  ? 

А) ништо, компајлерот ќе јави грешка    Б) ime
В) prezime                  Г) qrezime


6. Која од понудените можности е валиден повик на функцијата со име funkcija (под претпоставка дека таа постои и има 2 влезни параметри):

A) funkcija;                    Б) int funkcija(x,y);
В) funkcija x, y;               Г) funkcija(x,y);

7. Која од следниве дефиниции на функции е комплетна:

A) int funk(int a) {return a*а;}        
Б) void funk(char) { printf( "Zdravo");
В) void funk(y) { printf( "Zdravo"); } 
Г) int funk();

8. Со која од следниве наредби се декларира поле со 12 знаци:

А) int pole[12];                    Б) array char[12];
В) pole{12};                                            Г) char pole[12];

9. Со која од следниве наредби се пристапува до последниот елемент од полето primer ако истото е деклaрирано на следниов начин: char primer[25];

A) primer(24);                  Б) primer[25];
В) *(primer + 24);                         Г) primer(25);

10. Со која од следниве наредби се декларира целобројно дводимензионално поле  со max 30 елементи:

A) int pole[3*10];          Б) int pole[5][6];
В) int pole[30];            Г) int pole[6,5];

1. Што ќе се испечати по извршувањето на следниот програмски код: 
int j; char c; 
for(j=10,c='b';j;j-=2,c++);
 printf("%c\n",c);

a) кодот е невалиден            б) f        
в) g                        г)  l

2. int x=3, y=1;
printf(“%d %d”,--x, y++);

Што ќе се испечати на екран по извршувањето на горниот код?

а) 2 1          б) 2 2      в) 3 1          г) 3 2

3. Кој е валиден програмски код во програмскиот јазик С кој проверува дали ?  

а) if(10<x<=100)            б) if((100>x)&&(x>10))          
в) if((100>=x)&&(x>10))     г) while((100>=x)&&(x>10))

4. Кој од следниве изрази e валиден во програмскиот јазик C

а) for(x>0;x--) б) for(x=1)     в) while(10*10)     д) ниту еден

5. Што ќе се прикаже на екран по извршување на следниот програмски сегмент: 
int x = 5;
       if(-1<=x<1) printf("1");
else printf("0");

а) 0            б) 1        в) кодот не е валиден

6. Што ќе се отпечати на екран по извршување на следниов програмски код: 
#include<stdio.h>
int main ( )
{
    int a = 1; 
        for (printf ("100") ;printf ("") ;printf ("%d", a++)) {printf("100");break;}
}

       а) 100100             б) 1001001             в) 1001002              г) 100


7. Што ќе се испечати по извршување на следниот програмски код:
int a = 0; 
printf(" %d ", printf(" %d", printf("%d", a)));

a) 2 1 0        б) 0 1 2 3      в) 0 1 2            г) кодот е невалиден


8. int a = 1; 
switch(a)
{
             case 1:printf("3");
             case 2:printf("2");
             case 3:printf("1");
             default:printf("default");
}

      Што ќе испечати горната програма?

       а) 3      
б) 321      
в) 321default      
г) кодот е невалиден


9. int x = 4;
int y = 3/2 + 1/x + 1/2;
printf (“y=%03d\n”, y);

Што ќе се испечати ако се изврши горниот код? 

а) y=   1       б) y=2.25       в) y=1          г) y=001

10. Што ќе се прикаже на екран по извршување на следниов програмски сегмент: 

int x, y, temp, br =5;
x = 3; y = 5;
    do {
            temp = x; x = y; y = temp;
        br--;       
        }
    while (br);
    printf("%d %d %d\n",x,y,br);

a) 3 5 0            б) 5 3 0            в) 5 3 1    

1. Што ќе се испечати по извршување на следниот програмски код:

int main()
{   
    int niza[7]={1,2,3,4,5,6,7}, i = 0, n=7;
    *(niza+i)*=*(niza+n-1-i);
    printf("%d\n",*(niza));
    return 0;
}
а) 1        б) 7        в) 8        г) кодот не е валиден
2. Што ќе се испечати на екран по извршување на следниов програмски код:

int min (int a, int b, int c)
{ return(a < b ? a < c ? a : c : b ); }
int main()
{    
    printf("%d", min(3,4,5));
    system("PAUSE");
    return 0;
}
а) 5                б) 4            в) 3

3. Што ќе се испечати по извршувањето на следниот програмски код:
int a[10]={0,1,2,3},b[10]={1,2,3},c[10] = {1,2};
printf("%d\n",*(a + b[a[c[1]]]));
а) 0                    б) 2
в) 1                    г)  3

4. Со која од следниве наредби се читаат преку тастатура елементите од еднодимензионално поле со име a од 5 целобројни елементи:
а) for(i=0;i<5;i++)             б) scanf(“%d”,a);
scanf(“%d”,&a[i]);
в) for(i=0;i<5;i++)             г) for(i=1;i<=5;i++)    
    scanf(“%d”,a[i]);               scanf(“%d”,a[i]);

5. Што ќе се испечати на екран по извршување на следниот програмски код:
        int main()
        {   char c[13]={'j','a','s',’ ’,’s’,’u’,’m’,’ ’,'t','e','s','t','\0'}; 
        printf("%d",strlen(c)); 
return 0;       }

а) 12                   б) 13
в) 11                   г) jas sum test

6. Со која од следниве наредби се декларира дводимензионално поле (матрица) со димензија 100x100 (100 редици и 100 колони):

а) int matrica[100x100];        б) int matrica[100][100];
в) int matrica[100,100];        г) int matrica[10000];

7. Што ќе се испечати по извршување на следниот програмски код:
    int a=5, *b, *c;
    b = &a; c = b; *c=7;
    printf("%d\n",*b);

а) 5            б) 7            в) не може да се определи

8. Кој код отвора датотека ‘mojaDat.txt’во која ќе се запишува во C:

а) fopen("mojaDat.txt","write"));   б) fopen("mojaDat.txt","w"));
в) fopen("mojaDat.txt"));           г) fopen("mojaDat.txt","r"));

9. Кој е валиден код за вчитување на низа знаци во C:

а) scanf("%s",str);     б) getchar();
в) getstring(str);      г) ниту една од дадените наредби

10. Што ќе се испечати по извршувањето на следниот програмски код:

int f (int a)
{   printf("%d\t",a*a);
    if (a>1) f(a-2);
    printf("%d\t",a*a);    
}
int main()   {   f(7); return 0;  }

а) 49 25 9 1 1 9 25 49      б) 49 25 9 1
в) 1 9 25 49                г)  не може да се определи

1. Што ќе се испечати по извршување на следниот програмски код:

int main()
{   
    int niza[5]={1,2,3,4,5,6,7}, i = 0, n=7;
    *(niza+i)*=*(niza+n-1-i);
    printf("%d\n",*(niza));
    return 0;
}
а) 1        б) 7        в) 8        г) кодот не е валиден
2. Што ќе се испечати на екран по извршување на следниов програмски код:

int min (int a, int b, int c)
{ return(a < b ? a < c ? a : c : b ); }
int main()
{    
    printf("%d", min(3,4,5));
    system("PAUSE");
    return 0;
}
а) 5                б) 4            в) 3

3. Што ќе се испечати по извршувањето на следниот програмски код:
int a[10]={0,1,2,3},b[10]={1,2,3},c[10] = {1,2};
printf("%d\n",*(a + b[a[c[1]]]));
а) 0                    б) 2
в) 1                    г) 3

4. Со која од следниве наредби се читаат елементите од еднодимензионално поле со име a од 5 целобројни елементи:
а) for(i=0;i<5;i++)             б) scanf(“%d”,a);
scanf(“%d”,&a[10]);
в) for(i=0;i<5;i++)             г) for(i=1;i<=5;i++)    
    scanf(“%d”,a[i]);               scanf(“%d”,a[i]);
5. Што ќе се испечати на екран по извршување на следниот програмски код:
        int main()
        {   char c[13]={'j','a','s',’ ’,’s’,’um’,’ ’,'t','e','s','t','\0'}; 
        printf("%d",strlen(c)); 
return 0;       }

а) 12                   б) 13
в) 11                   г) ќе се јави грешка

6. Со која од следниве наредби се декларира дводимензионално поле (матрица) со димензија 100x100 (100 редици и 100 колони):

а) int matrica[100x100];        б) int matrica[100][100];
в) int matrica[100,100];        г) int matrica[10000];

7. Што ќе се испечати по извршување на следниот програмски код:
    int a=5, *b, *c;
    b = &a; c = b; *c=7;
    printf("%d\n",*b);

а) 5            б) 7            в) не може да се определи

8. Кој код отвора датотека ‘mojaDat.txt’во која ќе се запишува во C:

а) fopen("mojaDat.txt","write"));   б) fopen("mojaDat.txt","w"));
в) fopen("mojaDat.txt"));           г) fopen("mojaDat.txt","r"));

9. Кој е валиден код за вчитување на низа знаци во C:

а) scanf("%s",str);     б) getchar();
в) getstring(str);      г) ниту една од дадените наредби

10. Што ќе се испечати по извршувањето на следниот програмски код:

int f (int a)
{   printf("%d\n",a*a);
    if (a>1) f(a-2);
    printf("%d\n",a*a);    
}
int main()   {   f(7); return 0;  }

а) 49 25 9 1 1 9 25 49      б) 49 25 9 1
в) 1 9 25 49                г)  не може да се определи
