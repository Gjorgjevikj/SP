%%%%%%%%%%%%%%%%%%%%%%%%%%%%%%%%%%%%%%%%%
%%%%%%%%%% Content starts here %%%%%%%%%%
%%%%%%%%%%%%%%%%%%%%%%%%%%%%%%%%%%%%%%%%%

\section{Вектори (еднодимензионални полиња)}

\begin{frame}[fragile]{Задача 1}
Да се напише програма која за две низи кои се внесуваат од тастатура ќе провери
дали се еднакви или не. На екран да се испачати резултатот од споредбата.\\
Максимална големина на низите е 100.
\pause
\begin{exampleblock}{Решение 1 дел}
\lstinputlisting[lastline=11]{src/av5/p1.c}
\end{exampleblock}
\end{frame}

\begin{frame}[fragile]{Задача 1}{Решение 2 дел}
\begin{exampleblock}{Решение 2 дел}
\lstinputlisting[firstline=12]{src/av5/p1.c}
\end{exampleblock}
\end{frame}


\begin{frame}{Задача 2}
Да се напише програма која за низа, чии што елементи се внесуваат од стандарден влез, ќе го пресмета збирот на парните елементи, 
збирот на непарните елементи, како и односот помеѓу бројот на парни и непарни елементи. Резултатот да се испечати на екран.
\begin{exampleblock}{Пример}
За низата:\\
\texttt{3 {\color{red}2} 7 {\color{red}6} {\color{red}2} 5 1}\\
На екран ќе се испечати: \\
\texttt{suma\_parni = 8}\\
\texttt{suma\_neparni = 16}\\
\texttt{odnos = 0.75}
\end{exampleblock}
\end{frame}

\begin{frame}[fragile]{Задача 2}{Решение} 
\begin{exampleblock}{Решение}
\lstinputlisting{src/av5/p2.c}
\end{exampleblock}
\end{frame}

\begin{frame}{Задача 3}
Да се напише програма која ќе го пресмета скаларниот производ на два вектори со по n координати. 
Бројот на координати n, како и координатите на векторите се внесуваат од
стандарден влез. Резултатот да се испечати на екран.
\end{frame}

\begin{frame}[fragile]{Задача 3}{Решение} 
\begin{exampleblock}{Решение}
\lstinputlisting{src/av5/p3.c}
\end{exampleblock}
\end{frame}

\begin{frame}{Задачa 4}
Да се напише програма која ќе провери дали дадена низа од n елементи која се
внесува од тастатура е строго растечка, строго опаѓачка или ниту строго растечка
ниту строго опаѓачка. Резултатот да се испечати на екран.
\end{frame}

\begin{frame}[fragile,shrink=10]{Задача 4}{Решение} 
\begin{exampleblock}{Решение}
\lstinputlisting{src/av5/p4.c}
\end{exampleblock}
\end{frame}

\begin{frame}{Задача 5}
Да се напише програма која што ќе ги избрише дупликатите од една низа. На крај,
да се испечати на екран новодобиената низа. Елементите од низата се внесуваат од стандарден влез.
\end{frame}

\begin{frame}[fragile]{Задача 5}{Решение} 
\begin{exampleblock}{Решение}
\lstinputlisting{src/av5/p5.c}
\end{exampleblock}
\end{frame}

\section{Матрици (дводимензионални полиња)}

\begin{frame}{Задача 6}
Да се напише програма која ќе испечати на екран дали дадена матрица е симетрична во однос на главната дијагонала. 
Димензиите и елементите на матрицата се внесуваат од стандарден влез.
\end{frame}

\begin{frame}[fragile]{Задача 6}{Решение} 
\begin{exampleblock}{Решение}
\lstinputlisting{src/av5/p6.c}
\end{exampleblock}
\end{frame}


\begin{frame}{Задача 7}
Да се напише програма коjа за матрица внесена од тастатура ќе ги замени
елементите од главната диjагонала со разликата од максималниот и минималниот
елемент во матрицата. Резултантната матрица да се испечати на екран.
\end{frame}

\begin{frame}[fragile]{Задача 7}{Решение} 
\begin{exampleblock}{Решение}
\lstinputlisting{src/av5/p7.c}
\end{exampleblock}
\end{frame}

\begin{frame}{Задача 8}
Да се пресмета разликата на збирот на елементите во
непарните колони и збирот на елементите во парните редици.
Резултатот да се испечати на екран. Податоците за матрицата
се внесуваат од тастатура. Матрицата не мора да биде
квадратна.
\end{frame}

\begin{frame}[fragile]{Задача 8}{Решение} 
\begin{exampleblock}{Решение}
\lstinputlisting{src/av5/p8.c}
\end{exampleblock}
\end{frame}

\section{}

