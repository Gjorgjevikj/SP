
%%%%%%%%%%%%%%%%%%%%%%%%%%%%%%%%%%%%%%%%%
%%%%%%%%%% Content starts here %%%%%%%%%%
%%%%%%%%%%%%%%%%%%%%%%%%%%%%%%%%%%%%%%%%%


\begin{frame}[fragile]{Defining functions}{Remainders from lectures}
\begin{block}{Defining functions}
\begin{lstlisting}
return_type function_name(arguments_list) {
/* function body */
}
\end{lstlisting}
\end{block}
\begin{itemize}
    \item return\_type - type of the return value from the function
    \item function\_name - name of the function
    \item arguments\_list - the list of formal arguments contains all the
    arguments with thier coresponding types separated with coma, the body of the
    function contains the same elements as the \texttt{main()} function
\end{itemize}
\end{frame}

\begin{frame}[fragile]{Function calls}{Remainders from lectures}
\begin{block}{Function calls}
\begin{lstlisting}
function_name(arguments_list);
\end{lstlisting}
\end{block}
\begin{itemize}
    \item function\_name – the name of the function
    \item arguments\_list - the list of arguments is filled with values as the
    arguments, also separated with coma
\end{itemize}
\end{frame}

\begin{frame}[fragile]{Functions from the math library}{\texttt{math.h}}
\begin{itemize}
    \item C has standard math library \texttt{math.h} which contains many
    standard mathematical functions
    \item Before usage we should only include the library
    with \texttt{\#include<math.h>}
    \item All the functions from the standard library \texttt{math.h} work with
    arguments of type \texttt{double} and return values of the same type
\end{itemize}
\begin{block}{Including the math library}
\begin{lstlisting}
#include <math.h>
\end{lstlisting}
\end{block}
\end{frame}

\begin{frame}{Most commonly used math functions}{\texttt{math.h}}
\begin{tabular}{l | l}
\texttt{sqrt(x)} &  square root of $\sqrt{x}$\\
\hline
\texttt{exp(x)} & exponential function  $e^x$\\
\hline
\texttt{log(x)} & natural logarithm of $log(x)$ (with basis е)\\
\hline
\texttt{log10(x)} & logarithm of $x$ with basis 10 $log_{10}(x)$ \\
\hline
\texttt{fabs(x)} & absolute value of $|x|$\\
\hline
\texttt{ceil(x)} & round $\lceil{x}\rceil$ to the smallest whole number not
smaller than x\\
\hline
\texttt{floor(x)} & round $\lfloor{x}\rfloor$ to the largest whole number not
larger than x\\
\hline
\texttt{pow(x, y)}  & $x^y$\\
\hline
\texttt{fmod(x, y)}  & remainder x/y as a real number\\
\hline
\texttt{sin(x)} & $sin(x)$ (in radians) \\
\hline
\texttt{cos(x)} & $cos(x)$ (in radians)\\
\hline
\texttt{tan(x)} & $tan(x)$ (in radians)
\end{tabular}

\end{frame}

\begin{frame}{Problem 1}
Write a program that will print all fourdigit numbers divisible by the number
constructed from the sum of the first two digits and last two digits of the
number.
\begin{exampleblock}{Example}
\texttt{3417 is divisible with 34 + 17}\\
\texttt{5265 is divisible with 52 + 65}\\
\texttt{6578 is divisible with 65 + 78}
\end{exampleblock}
\end{frame}

\begin{frame}[fragile]{Problem 1}{Solution}
\lstinputlisting{src/av6/p1.c}
\end{frame}


\begin{frame}{Problem 2}
Write a program that for a given natural number will compute the difference
between that number and the following prime number.
\begin{exampleblock}{Example}
For the number \texttt{573}, the program should print\\
\texttt{577 – 573 = 4}
\end{exampleblock}
\end{frame}

\begin{frame}[fragile]{Problem 2}{Solution} 
\lstinputlisting{src/av6/p2.c}
\end{frame}

\begin{frame}{Problem 3}
Write a program that will print all primes smaller than 10000 whose sum of
digits is also a prime. At the end print the count of that kind of numbers.
\begin{exampleblock}{Example}
\texttt{23 -> 2+3=5}\\
\texttt{179 -> 1+7+9=17}\\
\texttt{9613 -> 9+6+1+3=19}
\end{exampleblock}
\end{frame}

\begin{frame}[fragile]{Problem 3}{Solution} 
\begin{columns}
    \column{.5\textwidth}
    \lstinputlisting[lastline=25]{src/av6/p3.c}
    \column{.5\textwidth}
    \lstinputlisting[firstline=26]{src/av6/p3.c}
\end{columns}
\end{frame}

\begin{frame}{Problem 4}
Write a program that will print all pairs of primes up to 1000 that
differentiate between themselves for 2. At the end print the count.
\begin{exampleblock}{Example}
\texttt{11 and 13}\\
\texttt{101 and 103}\\
\texttt{617 and 619}\\
\texttt{881 and 883}
\end{exampleblock}
\end{frame}

\begin{frame}[fragile]{Problem 4}{Solution}
   \lstinputlisting{src/av6/p4.c}
\end{frame}

\begin{frame}{Problem 5}
Compute the sum:\\
\texttt{1!+(1+2)!+(1+2+3)!+\ldots+(1+2+...+n)!}
\\Help:\\
\begin{itemize}
    \item Use function for computing the sum of the first k natural numbers
    \item Use a function for computing factorial of k
\end{itemize}
\end{frame}

\begin{frame}[fragile]{Problem 5}{Solution}
   \lstinputlisting{src/av6/p5.c}
\end{frame}

\begin{frame}{Problem 6}
Write a function that will compute following expression for $x$ and $n$:
\[
   f(x, n) = \left\{ 
  \begin{array}{l l}
    x + \frac{x^n}{n} + \frac{x^{n+2}}{n + 2} &,x\geq0\\[10px]
    - \frac{x^{n - 1}}{n - 1} + \frac{x^{n+1}}{n + 1} &,x<0
  \end{array} \right.
\]
Than write a program that will table this function for read n in interval
$x\in[-4, 4]$, with step $0.1$.
\end{frame}

\begin{frame}[fragile]{Problem 6}{Solution}
\begin{columns}
    \column{.5\textwidth}
          \lstinputlisting[lastline=21]{src/av6/p6.c}
    \column{.5\textwidth}
        \lstinputlisting[firstline=22]{src/av6/p6.c}
\end{columns}
\end{frame}

