
%%%%%%%%%%%%%%%%%%%%%%%%%%%%%%%%%%%%%%%%%
%%%%%%%%%% Content starts here %%%%%%%%%%
%%%%%%%%%%%%%%%%%%%%%%%%%%%%%%%%%%%%%%%%%


\begin{frame}[fragile]{Recursive functions}{Remainders from lectures}
\begin{itemize}
    \item Functions in C can call other functions
    \begin{itemize}
        \item This calling can go in some limited depth
    \end{itemize}    
    \item A function can call itself
    \item That way of calling function is called \textbf{recursion}    
\end{itemize}
\begin{exampleblock}{Example of recursive function}
\begin{lstlisting}
int faktorial(int n) {
    if(n == 0) return 1;
    else return n * faktorial(n - 1);
}
\end{lstlisting}
\end{exampleblock}
\end{frame}

\begin{frame}{Problem 1}
Compute the sum:\\
\texttt{1!+(1+2)!+(1+2+3)!+\ldots+(1+2+...+n)!}
\\This time:\\
\begin{itemize}
    \item Use \textbf{recursive} function to compute sum of the first
    k numbers
    \item Use \textbf{recursive} function to compute factorial of a number k
\end{itemize}
\end{frame}

\begin{frame}[fragile]{Problem 1}{Solution}
\lstinputlisting[lastline=14]{src/av7/p1.c}
\end{frame}

\begin{frame}[fragile]{Problem 1}{Solution}
\lstinputlisting[firstline=15]{src/av7/p1.c}
\end{frame}


\begin{frame}{Problem 2}
Write a program that for a given natural number will compute the difference
between that number and the following prime number. The program should use
\textbf{recursive} function to check if number is prime.
\begin{exampleblock}{Example}
For the number \texttt{573}, the program should print\\
\texttt{577 – 573 = 4}
\end{exampleblock}
\end{frame}

\begin{frame}[fragile]{Problem 2}{Solution} 
\lstinputlisting{src/av7/p2.c}
\end{frame}

\begin{frame}{Problem 3}
Write a function that will return the value of n-th member of the sequence
defined with:
\[
   \begin{array}{l}
   x_1 = 1\\
   x_2 = 2\\ 
   \vdots\\
   x_n = \frac{n - 1}{n}x_{n - 1} + \frac{1}{n}x_{n - 2}
   \end{array}
\]
\end{frame}


\begin{frame}[fragile]{Problem 3}{Solution}
\lstinputlisting{src/av7/p3.c}
\end{frame}


\begin{frame}[fragile]{Problem 4}
Write a recursive function that will compute the sum of the digits of a
given number.
\begin{exampleblock}{Example}
\begin{verbatim}
sumDigits(126) -> 9
sumDigits(49) -> 13
sumDigits(12) -> 3
\end{verbatim}
\end{exampleblock}
\pause
\lstinputlisting{src/av7/p4.c}
\end{frame}

\begin{frame}[fragile]{Problem 5}
Given a non-negative int n, compute recursively (no loops) the count of the
occurrences of 8 as a digit, except that an 8 with another 8 immediately to its
left counts double, so 8818 yields 4.
\begin{exampleblock}{Example}
\begin{verbatim}
count8(8) -> 1
count8(818) -> 2
count8(8818) -> 4
\end{verbatim}
\end{exampleblock}
\pause
\lstinputlisting{src/av7/p5.c}
\end{frame}


\begin{frame}{Problem 6}
Write a program that for given array of integers (read from SI) will print the
greatest common divisor (GCD) of its elements. GCD should be computed using
recursive function.
\begin{exampleblock}{Example}
\texttt{48 36 120 72 84}\\
Should print:\\
\texttt{GCD is 12}
\end{exampleblock}
\end{frame}

\begin{frame}[fragile,shrink=5]{Euclidean algorithm}{Problem 6}
\begin{itemize}
  \item GCD for two numbers can be computed using the Euclidean algorithm
  \item To compute GCD of numbers m and n, we compute the remainder of
  division of m with n
  \begin{itemize}
  \item If remainder is not 0, we compute the remainder of division of n with (m
  \% n)
  \item This step is repeated until the remainder is zero.
  \item If the remainder is 0, GCD of the two numbers is the last non zero
  remainder.
  \end{itemize}
\end{itemize}
\begin{exampleblock}{Example}
\begin{verbatim}
GCD(20, 12)
20 % 12 = 8
12 % 8 = 4
8 % 4 = 0
GCD(20, 12) = 4
\end{verbatim}
\end{exampleblock}
\end{frame}

\begin{frame}[fragile]{Problem 6}{Solution}
\lstinputlisting{src/av7/p6.c}
\end{frame}

\begin{frame}[fragile]{Problem 7}{Try to solve at home}
Write a program that for given array of natural numbers (read from SI) will find
and print the least common denominator (LCD) of its elements. The program should
use recursive function for computing LCD of two numbers.
\begin{exampleblock}{Example}
\begin{verbatim}
For array:
18 12 24 36 6
The program should print:
LCD is 72
\end{verbatim}
\end{exampleblock}
\end{frame}
   

\begin{frame}[fragile]{Problem 8}
Write a program that fro given array of integers (read from SI) will print the
smallest element. The program should use recursive function for finding the
smallest element of an array.
\begin{exampleblock}{Example}
\begin{verbatim}
For array:
5 8 3 12 9 6
The result should be:
3
\end{verbatim}
\end{exampleblock}
\end{frame}

\begin{frame}[fragile]{Problem 8}{Solution}
\lstinputlisting{src/av7/p8.c}
\end{frame}

