%%%%%%%%%%%%%%%%%%%%%%%%%%%%%%%%%%%%%%%%%
%%%%%%%%%% Content starts here %%%%%%%%%%
%%%%%%%%%%%%%%%%%%%%%%%%%%%%%%%%%%%%%%%%%


\begin{frame}[fragile,shrink=10]{Pointers}{Remainders from lectures}
  \begin{block}{What is a pointer?}
    \textbf{Pointer} is \alert{data type} that stores (points to) some memory
    location. This memory location is stored by its address (some number).
  \end{block}
  \begin{block}{How to declare pointer?}
  
    \texttt{pointer\_type *name;}\\
    Example:\\
    \texttt{float *p;}\\
    \alert{This pointer doesn't point to anything!}
  \end{block}
  \begin{block}{Why do we use pointers?}
    \begin{itemize}
      \item For fast and efficient looping of complex data structures as arrays
      and trees, \ldots
      \item For efficient passing of complex arguments in functions. Passing
      array, struct or similar.
      \item For passing arguments whose values we want to stay affected after
      the function call.
    \end{itemize}
  \end{block}
  
\end{frame}

\begin{frame}{Problem 1}
Write a function that for an array of N integers will find the start and length
of the largest increasing subarray.
\begin{exampleblock}{Example}
For array\\
\texttt{2 3 {\color{red}1 4 7 12} 7 9 1}\\
will return \texttt{2 4}
\end{exampleblock}
\end{frame}

\begin{frame}[fragile]{Problem 1}{Solution}
\begin{columns}
\column{.5\textwidth}
\lstinputlisting[lastline=23]{src/av8/p1.c}
\column{.5\textwidth}
\lstinputlisting[firstline=25]{src/av8/p1.c}
\end{columns}
\end{frame}

\begin{frame}{Problem 2}
Write a function that for the input array $a_0, a_1, \ldots, a_{n-1}$ will
transform in to the output array $b_0, b_1,
\ldots, b_{n-1}$ on the following way
\[
\begin{array}{l}
b_0 = a_0 + a_{n-1}\\
b_1 = a_1 + a_{n-2}\\
\vdots\\
b_n = a_{n-1} + a_{0}
\end{array}
\]
\begin{exampleblock}{Example} 
Input array\\
\texttt{1 2 3 5 7}\\
should be transformed in\\
\texttt{8 7 6 7 8}
\end{exampleblock}
\end{frame}

\begin{frame}[fragile]{Problem 2}{Solution} 
\lstinputlisting{src/av8/p2.c}
\end{frame}

\begin{frame}{Problem 3}
Write the following functions for searching an array:
\begin{itemize}
  \item Linear search
  \item Binary search
\end{itemize}  
Then write a program that fill an array with numbers from 1 to 1000000,
and than generates random number in this range and finds its index by calling
the two functions for searching.\\
\textbf{Homework:} For the functions count and compare the number of needed
iterations to find the number.
\end{frame}


\begin{frame}[fragile]{Problem 3}{Solution}
\lstinputlisting[firstline=6,lastline=23]{src/av8/p3.c}
\textbf{Homework:} Write a recursive function for binary search.
\end{frame}

\begin{frame}[fragile]{Problem 3}{Solution}
\lstinputlisting[firstline=26]{src/av8/p3.c}
\end{frame}


\begin{frame}{Problem 4}
Write functions for sorting array by using the following methods for sorting:
\begin{itemize}
  \item Bubble sort
  \item Selection sort
  \item Insertion sort
\end{itemize}
Write functions fro reading and printing elements of an array, and write main
program to test the sort functions.
\end{frame}

\begin{frame}[fragile]{Bubble sort}{Problem 4}
\begin{scriptsize}
The method starts with comparing the first element with its neighbour until we
reach the last element. In each comparison, if the previous has larger value,
then they swap their values. That way the largest element is moved to the last
position in the array. This method is repeated from 1-st to the last but one
element in the array, so this position will float the element that is largest
but smaller than the last element in the array. At the and we compare only the
1-st and 2-nd element of the array.
\end{scriptsize}
\lstinputlisting[firstline=3, lastline=11]{src/av8/p4.c}
\end{frame}

\begin{frame}[fragile]{Selection sort}{Problem 4}
\begin{scriptsize}
We find the smallest element of the array and we swap the value of this element
with the first element. After that, the first element is ignored (since its
smallest) and recursively the method is repeated to rest of the array (second
element, until the end). The recursion stops when there is only one element
left.
That is the final step, and the sorting is done.
\end{scriptsize}
\lstinputlisting[firstline=13, lastline=28]{src/av8/p4.c}
\end{frame}

\begin{frame}[fragile]{Insertion sort}{Problem 4}
\begin{scriptsize}
With this method we sort by inserting each element in appropriate position,
hence the name of the sort. After the first iteration, the second element a[1] is
compared to the first element a[0]. After the second iteration the third
element is compared with first and the second. In general, in each iteration
the element is compared with all elements in front of it. If the comparison
shows that the element should be inserted at some position, then we make space
by shifting all elements one place in right. This procedure is repeated for
each element in the array.
\end{scriptsize}
\lstinputlisting[firstline=30, lastline=41]{src/av8/p4.c}
\end{frame}

\begin{frame}[fragile]{Problem 4}{Solution}
\lstinputlisting[firstline=43]{src/av8/p4.c}
\end{frame}
