\documentclass[11pt]{examdesign}
%\usepackage{ucs}
\usepackage[T2A]{fontenc}
\usepackage[utf8]{inputenc}
\usepackage{amsmath}
\usepackage{pifont}
\usepackage{verbatim}
\usepackage[ddmmyyyy]{datetime}
\renewcommand{\dateseparator}{.}
\SectionFont{\large\sffamily}
\usepackage[margin=1.5cm]{geometry}
%\Fullpages
\ContinuousNumbering
%\ShortKey
\DefineAnswerWrapper{}{}
\NumberOfVersions{1}
\IncludeFromFile{sp_test2_code.tex}

\def\namedata{Name: \hrulefill \\[5pt]
ID: \hrulefill}

\begin{examtop}
{\parbox{.5\textwidth}{\textbf{\classdata} \\
\examtype, 17.12.2014, Group: \fbox{\textsf{A}}\\ \emph{Each question
has exactly \textbf{one correct} answer.}}
\hfill
\parbox{.45\textwidth}{\normalsize \namedata}}
\end{examtop}

\def\aftersectsep{0pt}
\def\beforesectsep{0pt}
\def\beforeinstsep{0pt}
\def\afterinstsep{0pt}


\class{\Large{Structured Programming}}
\examname{Test 2}
\begin{document}
%\SectionPrefix{Дел \arabic{sectionindex}. \space}


\begin{multiplechoice}[title={},suppressprefix=yes,rearrange=no]
\begin{question}
Which of the following declaration is \textbf{not} a valid function prototype:
    \choice{\texttt{int func(char,char);}}
    \choice[!]{\texttt{double funcion;}}
    \choice{\texttt{void func();}}
    \choice{\texttt{float funcion();}}
\end{question}

\begin{question}
What will be the output after execution of the following code segment?
\InsertChunk{c1A}
    \choice{nothing, it's compilation error}
    \choice[!]{\texttt{ime}}
    \choice{\texttt{prezime}}
    \choice{\texttt{qrezime}}
\end{question}

\begin{question}
What will be the output after execution of the following code segment?
  \InsertChunk{c2A}
  \choice{\texttt{1}}
  \choice{\texttt{3}}
  \choice[!]{\texttt{5}}
  \choice{can not be determined}
\end{question}

\begin{question}
Which is the correct expression for accessing the last element of the array
defined as: \texttt{int elements[] = \{ 5, 4, 3, 2, 1 \};}
    \choice{\texttt{*(elements + 4 * sizeof(int))}}               
    \choice{\texttt{elements + 5}}
    \choice[!]{\texttt{*(elements + 4)}}
    \choice{\texttt{elements[5]}}  
\end{question}

\begin{question}
What will be the value of \texttt{a} after the execution of the following code segment?
    \InsertChunk{c3A}
    \choice[!]{\texttt{5}}
    \choice{\texttt{4}}
    \choice{\texttt{3}}
    \choice{undefined}
\end{question}
  
\begin{question}
What will be the value of \texttt{len} after the execution of the following code
segment?
    \InsertChunk{c4A}
    \choice[!]{\texttt{5}}
    \choice{\texttt{3}}
    \choice{\texttt{4}}
    \choice{undetermined}
\end{question}
  
\begin{question}
Which is the correct expression in C to open a file named \texttt{"in.dat"} in
readonly mode?
    \choice{\texttt{fopen("in.txt","rb");}}
    \choice{\texttt{fopen("in.dat","w");}}
    \choice{\texttt{fopen("in.dat", "read");}}
    \choice[!]{\texttt{fopen("in.dat", "r");}}
\end{question}

\begin{question}
Which of the following functions is used to write a character in a file in C?
    \choice{\texttt{void writef(file f)}}
    \choice[!]{\texttt{int fputc(int character, FILE *f);}}
    \choice{\texttt{void fileput(int c, F* file);}}
    \choice{\texttt{int putchar(int character);}}
\end{question}

\begin{question}
What will be the output after execution of the following code segment?
\InsertChunk{c6A}
    \choice[!]{\texttt{49 25 9 1 1 9 25 49}}
    \choice{49 25 9 1}
    \choice{1 9 25 49}
    \choice{could not be determined}
\end{question}
  
\begin{question}
What will be the output after execution of the following code segment?
\InsertChunk{c7A}
\choice{\texttt{0}}
\choice{\texttt{1}}
\choice{\texttt{2}}
\choice[!]{\texttt{3}}
\end{question}

\end{multiplechoice}

\end{document}
