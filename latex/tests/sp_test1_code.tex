
\begin{chunk}{c0}
\begin{verbatim}
int a = 0; 
printf(" %d ", printf(" %d", printf("%d", a)));
\end{verbatim}
\end{chunk}

\begin{chunk}{c1}
\begin{verbatim}
int x = 4;
int y = 3/2 + 1/x + 1/2;
printf ("y = %03d\n", y);
\end{verbatim}
\end{chunk}

\begin{chunk}{c1B}
\begin{verbatim}
int a = 4;
int b = 1/a + 3/2 + 1/2;
printf ("b = %03d\n", b);
\end{verbatim}
\end{chunk}

\begin{chunk}{c2}
\begin{verbatim}
int x; for(x = 10; x >= 0; x--) {} 
\end{verbatim}
\end{chunk}

\begin{chunk}{c2B}
\begin{verbatim}
int i; for(i = 0; i <= 10; ++i) {} 
\end{verbatim}
\end{chunk}

\begin{chunk}{c3}
\begin{verbatim}
int a, b, d = 2; float c = 0;
for(a = 5, b = a--; a > 0, b < 10; a--, b++) c += 1 / d;
printf("%3.1f\n", c);
\end{verbatim}
\end{chunk}

\begin{chunk}{c4}
\begin{verbatim}
if(1 <= x <= 2) printf("YES");
else printf("NO");
\end{verbatim}
\end{chunk}

\begin{chunk}{c4B}
\begin{verbatim}
if(-1 < x < 0) printf("YES");
else printf("NO");
\end{verbatim}
\end{chunk}


\begin{chunk}{c5}
\begin{verbatim}
int x = 23;
switch(x) {
  case 1: printf("1"); break;
  case 23: printf("2"); break;
  case 123: printf("3"); break;
}
\end{verbatim}
\end{chunk}

\begin{chunk}{c5B}
\begin{verbatim}
int a = 1;
switch(a) {
  case 1: printf("1");
  case 12: printf("2");
  case 123: printf("3");
}
\end{verbatim}
\end{chunk}

\begin{chunk}{c6}
\begin{verbatim}
int x; for(x = 0; x <= 10; ++x){}
\end{verbatim}
\end{chunk}

