\documentclass[11pt]{examdesign}
\usepackage{ucs}
\usepackage[T2A]{fontenc}
\usepackage[utf8]{inputenc}
\usepackage{amsmath}
\usepackage{pifont}
\usepackage{verbatim}
\usepackage[ddmmyyyy]{datetime}
\renewcommand{\dateseparator}{.}
\SectionFont{\large\sffamily}
\usepackage[margin=1.5cm]{geometry}
%\Fullpages
\ContinuousNumbering
%\ShortKey
\DefineAnswerWrapper{}{}
\NumberOfVersions{1}
\IncludeFromFile{sp_test1_code.tex}

\def\namedata{Name: \hrulefill \\[5pt]
ID: \hrulefill}

\begin{examtop}
{\parbox{.5\textwidth}{\textbf{\classdata} \\
\examtype, 29.10.2014, Group: \fbox{\textsf{A}}\\ \emph{Each question
has exactly \textbf{one correct} answer.}}
\hfill
\parbox{.45\textwidth}{\normalsize \namedata}}
\end{examtop}

\def\aftersectsep{0pt}
\def\beforesectsep{0pt}
\def\beforeinstsep{0pt}
\def\afterinstsep{0pt}


\class{\Large{Structured Programming}}
\examname{Test 1}
\begin{document}
%\SectionPrefix{Дел \arabic{sectionindex}. \space}


\begin{multiplechoice}[title={},suppressprefix=yes,rearrange=no]
\begin{question}
What is the valid code in C that checks the following condition 10 < x <
100?
    \choice{\texttt{if(10 < x <= 100)}}
    \choice{\texttt{if((100 > x) \&\& (x > 10))}}
    \choice[!]{\texttt{if((100 >= x) \&\& (x > 10))}}
    \choice{\texttt{while((100 >= x) \&\& (x > 10))}}
\end{question}

\begin{question}
Which of the following expressions \textbf{is valid} in C?
    \choice[!]{\texttt{while(10 * 10)}}
    \choice{\texttt{for(x > 0; x--)}}
    \choice{\texttt{for(x = 1)}}
    \choice{neither}
\end{question}

\begin{question}
What will be the output after execution of the following code segment?
  \InsertChunk{c0}
  \choice{\texttt{2 1 0}}
  \choice{\texttt{0 1 2 3}}
  \choice[!]{\texttt{0 1 2}}
  \choice{the code is invalid}
\end{question}

\begin{question}
What will be the output after execution of the following code segment?
    \InsertChunk{c1}
    \choice{\texttt{y = 1}}               
    \choice{\texttt{y = 2.25}}
    \choice{\texttt{y = 1.00}}
    \choice[!]{\texttt{y = 001}}  
\end{question}

\begin{question}
What will be the value of x after the execution of the following code segment?
    \InsertChunk{c2}
    \choice{\texttt{0}}
    \choice{\texttt{1}}
    \choice[!]{\texttt{-1}}
    \choice{\texttt{9}}
\end{question}
  
\begin{question}
Which of the following expressions will declare an array of 5 integers?
    \choice{\texttt{pole int[5];}}
    \choice{\texttt{int pole[];}}
    \choice[!]{\texttt{int pole[] = \{1, 2, 3, 4, 5\};}}
    \choice{\texttt{array int[5];}}
\end{question}
  
\begin{question}
What will be the output after the execution of the following code segment?
    \InsertChunk{c5}
    \choice{\texttt{123}}
    \choice{\texttt{1}}
    \choice{\texttt{23}}
    \choice[!]{\texttt{2}}
\end{question}
  
\begin{question}
What is the result from the execution of the following code segment?
    \InsertChunk{c3}
    \choice{2.5}
    \choice[!]{0.0}
    \choice{2.0}
    \choice{3.0}
\end{question}

\begin{question}
What will be the output after the execution of the following code segment?
    \InsertChunk{c4}
    \choice[!]{YES}
    \choice{NO}
    \choice{cannot predict, will depend on value of variable x}
    \choice{the code will produce a syntax error}
  \end{question}

\begin{question}
Which from the following expressions does not result in 3.5 if the variables are declared as \texttt{int a=7, b=2;}
  \choice{\texttt{a/(float)b}}
  \choice{\texttt{a*1./b}}
  \choice{\texttt{float(a)/b}}
  \choice[!]{\texttt{(float)(a/b)}}
\end{question}

\end{multiplechoice}

\end{document}
