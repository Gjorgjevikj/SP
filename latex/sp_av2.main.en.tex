%%%%%%%%%%%%%%%%%%%%%%%%%%%%%%%%%%%%%%%%%
%%%%%%%%%% Content starts here %%%%%%%%%%
%%%%%%%%%%%%%%%%%%%%%%%%%%%%%%%%%%%%%%%%%


\begin{frame}{Reminding From Lecture}
\begin{itemize}
	\item Operators
	\begin{itemize}
		\item Arithmetic
		\item Relational
		\item Logical
	\end{itemize}
	\item Printing to standard output
	\texttt{printf()}
	\item Reading from standard input
	\texttt{scanf()}
\end{itemize}
\end{frame}

\begin{frame}[fragile]{Reading from standard input in C}{Function
\texttt{scanf}}
	\begin{verbatim}
	int scanf(control_char_array, arg1, arg2, ..., argn)	
	\end{verbatim}	
	\begin{itemize}
	\item The control char array actualy contains nessesry information
	for formatting
	\item arg1, arg2, ..., argn are arguments for individual data type
	\end{itemize}	 
\end{frame}

\begin{frame}[fragile]{Using \texttt{scanf}}

	\begin{exampleblock}{Example 1}
	\begin{lstlisting}
	#include <stdio.h>
	int main() {
	    char c;
	    int number;
	    float price;
	    scanf("%c%d%f", &c, &number, &price);
	    return 0;
	}
	\end{lstlisting}
	\end{exampleblock}

\end{frame}

\begin{frame}[fragile]{Problem 1}
Write a program for computing and printing the circle area and perimeter. The
circle radius is read as decimal number.
\begin{exampleblock}{Solution}
\lstinputlisting{src/av2/p1.c}
\end{exampleblock}
\end{frame}


\begin{frame}[fragile]{Problem 2}
Write a program that reads from standard input two integers and prints their
sum, difference, product and division remainder.
	\begin{exampleblock}{Solution}
		\lstinputlisting{src/av2/p2.c}
	\end{exampleblock}
\end{frame}

\begin{frame}[fragile]{Problem 3}
Write a program that reads uppercase letter from standard input and prints out
in lowercase.\\ Help: Each character is represented with its ASCII code.\\
Ex. \texttt{'А' = 65, 'а' = 97}
	\begin{exampleblock}{Solution}
		\lstinputlisting{src/av2/p3.c}
	\end{exampleblock}
\end{frame}

\begin{frame}[fragile]{Problem 4}
\begin{scriptsize}
Write a program that reads a character from standard input, and prints 1 if it's
lowercase or 0 if it's uppercase.\\ Help: Use logical and relational
operators to toest ASCII code of the character.\\
\textbf{Extra:} Check if the character is digit.
\end{scriptsize}
\begin{exampleblock}{Solution}
	\lstinputlisting{src/av2/p4.c}		
\end{exampleblock}
\end{frame}

\begin{frame}[fragile]{Problem 5}
\begin{scriptsize}
Write a program that reads to integers (x, y) from standard input and prints on
the standard output the result (z) of the following expression\\ \texttt{z
= x++ + ---y + (x<y)}\\ What is the value of z for x=1, y=2?
\end{scriptsize}
	\begin{exampleblock}{Solution}
		\lstinputlisting{src/av2/p5.c}
	\end{exampleblock}
\end{frame}

\begin{frame}[fragile]{Problem 6}
Let \texttt{r = (x < y || y < z++)}\\
	What is the value of r for x=1, y=2, z=3?\\
	What is the value of z?
	\begin{exampleblock}{Solution}
	\texttt{r = 1\\z = 3}
	\end{exampleblock}
Let: \texttt{r = (x > y \&\& y < z++)}\\
	What is the value of r for x=1, y=2, z=3?\\
	What is the value of z?
	\begin{exampleblock}{Solution}
	\texttt{r = 0\\z = 3}
	\end{exampleblock}
\end{frame}

\begin{frame}[fragile]{Problem 7}
Let:
\begin{lstlisting}
	int x, y;
	y = scanf("%d", &x);
\end{lstlisting}
What is the value of y for x=5?
	\begin{exampleblock}{Solution}
	\texttt{y = 1}
	\end{exampleblock}
Let:
\begin{lstlisting}
	int x, y, z;
	z = scanf("%d%d", &x, &y);
\end{lstlisting}
	What is the value of z for x=5, y=6?
	\begin{exampleblock}{Solution}
	\texttt{z = 2}
	\end{exampleblock}
\end{frame}


\begin{frame}[fragile]{Problem 8}
Write a program that reads from SI product price and then will print the
price with calculated sales tax.\\ Help: Sales tax is 18\% of the starting
price.
	\begin{exampleblock}{Solution}
		\lstinputlisting{src/av2/p8.c}
	\end{exampleblock}
\end{frame}


\begin{frame}[fragile]{Problem 9}
Write a program that reads from SI product price, payments number and interest
rate (interest rate is percentage number from 0 to 100). The program should
print the payments value and the total sum for the product.\\
Help: First calculate the total sum, then the payment.
\end{frame}

\begin{frame}[fragile]{Problem 9}{Solution}
	\lstinputlisting{src/av2/p9.c}
\end{frame}


\begin{frame}[fragile]{Problem 10}
Write a program that reads from SI one tri digit number. The program should
print the most and least significant digit.\\ 
Example: For the number 795, the output should be:\\
\texttt{	Most significant digit is 7, and least significant is 5.}\\
Help: Use division and modulo operation.
\end{frame}

\begin{frame}[fragile]{Problem 10}{Solution}
\lstinputlisting{src/av2/p10.c}
\end{frame}


\begin{frame}[fragile]{Problem 11}
Write a program that reads from SI the birth date in format (\texttt{ddmmgggg}).
The program should print the day and month of birth.\\
Example: For the following number \texttt{18091992}, the program should print:
\texttt{18.09}\\ Help: Use division and module operators.
\end{frame}

\begin{frame}[fragile]{Problem 11}{Solution}
	\begin{exampleblock}{Solution}
	\lstinputlisting{src/av2/p11.c}	
	\end{exampleblock}
	Extra: The problem can be solved by using
	\texttt{scanf("\%2d\%2d", \&den, \&mesec)}
\end{frame}
