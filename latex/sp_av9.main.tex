%%%%%%%%%%%%%%%%%%%%%%%%%%%%%%%%%%%%%%%%%
%%%%%%%%%% Content starts here %%%%%%%%%%
%%%%%%%%%%%%%%%%%%%%%%%%%%%%%%%%%%%%%%%%%


\begin{frame}{Функции за работа со текстуални
низи}{\texttt{<string.h>}}
Функции за менување на текстуални низи
\begin{itemize}
  \item \texttt{strcpy} - копирање на една текстуална низа во друга
  \item \texttt{strncpy} - копирање на n бајти во тесктуална низа, се
  копираат од \texttt{src} или се додаваат \texttt{nulls}
  \item \texttt{strcat} - додава една текстуална низа на крајот на друга
  \item \texttt{strncat} - додава n бајти од една текстуална низа во друга
\end{itemize}
Функции за менување на меморијата
\begin{itemize}
  \item \texttt{memset} - пополнува низа со одреден бајт
\end{itemize}
Функции за претворање на низи од знаци во броеви
\begin{itemize}
  \item \texttt{atof} - претвора низа од знаци во децимален број
  \item \texttt{atoi} - претовра низа од знаци во цел број 
\end{itemize}

\end{frame}

\begin{frame}[shrink=10]{Функции за работа со текстуални
низи}{\texttt{<string.h>}}
Функции за испитување на текстуални низи
\begin{itemize}
  \item \texttt{strlen} - ја враќа должината на низата
  \item \texttt{strcmp} - споредува две текстуални низи
  \item \texttt{strncmp} - споредува одреден број бајти од две текстуални низи
  \item \texttt{strchr} - го наоѓа првото појавување на даден знак во текстуална низа
  \item \texttt{strrchr} - го наоѓа последното појавување на знак во низа
  \item \texttt{strspn} - го наоѓа првото појавување во текстуална низа на знак кој
  не е во множество од знаци
  \item \texttt{strcspn} - го наоѓа последното појавување во текстуална низа на знак кој
  не е во множество од знаци
  \item \texttt{strpbrk} - го наоѓа првото појавување во текстуална низа на знак од
  множество од знаци
  \item \texttt{strstr} - го наоѓа во текстуална низа првото појавување на подниза
  \item \texttt{strtok} - го наоѓа во текстуална низа следното појавување на токен
\end{itemize} 
\end{frame}

\begin{frame}[shrink=10]{Функции за работа со знаци}{\texttt{<ctype.h>}}
\begin{itemize}
  \item \texttt{isalnum} - проверува дали даден знак е алфанумерички (буква или
  број)
  \item \texttt{isalpha} - проверува дали даден знак е буква
  \item \texttt{iscntrl} - проверува дали даден знак е контролен знак
  \item \texttt{isdigit} - проверува дали даден знак е декадна цифра
  \item \texttt{isxdigit} - проверува дали даден знак е хексадецимална цифра
  \item \texttt{isprint} - проверува дали знакот може да се печати
  \item \texttt{ispunct} - проверува дали даден знак е интерпукциски
  \item \texttt{isspace} - проверува дали дали даден знак е празно место
  \item \texttt{islower} - проверува дали даден знак е мала буква
  \item \texttt{isupper} - проверува дали даден знак е голема буква
  \item \texttt{tolower} - претвора голема во мала буква
  \item \texttt{toupper}  - претвора мала во голема буква
  \item \texttt{isgraph} - проверува дали даден знак има локална графичка
  репрезентација
\end{itemize}
\end{frame}




\begin{frame}{Задачa 1}
Да се напише функција што ќе одреди колку пати знак се наоѓа во даден стринг.
Знакот за споредување и стрингот се внесуваат од тастатура.
\begin{exampleblock}{Пример}
За стрингот\\
\texttt{``hello FINKI''}\\
знакот \texttt{'l'} се наоѓа 2 пати
\end{exampleblock}
\end{frame}

\begin{frame}[fragile]{Задача 1}{Решение}
\lstinputlisting{src/av9/p1.c}
\end{frame}

\begin{frame}{Задачa 2}
Да се напише функција што ќе ја одреди должината на една текстуална низа. Да се
понуди и рекурзивно решение.
\begin{exampleblock}{Пример} 
Ако се внесе: \texttt{``zdravo!''}\\
Треба да врати: 7
\end{exampleblock}
\end{frame}

\begin{frame}[fragile]{Задачa 2}{Решение} 
\lstinputlisting{src/av9/p2.c}
\end{frame}

\begin{frame}{Задачa 3}
    Да се напишe програма која ќе врати подниза од зададена текстуална низа
    определена со позицијата и должината што како параметри се вчитуваат од
    тастатура. Поднизата започнува од карактерот што се наоѓа на соодветната
    позиција во текстуалната низа броено од лево. 
\begin{exampleblock}{Пример}
Ако се внесе:\\
\texttt{``banana''}, позиција = 3, должина = 4\\
Треба да се добие: \texttt{nana}
\end{exampleblock}    
\end{frame}


\begin{frame}[fragile]{Задачa 3}{Решение}
\lstinputlisting{src/av9/p3.c}
\end{frame}

\begin{frame}{Задачa 4}
Да се напише функција која ќе одреди дали една текстуална низа е подниза на
друга текстуална низа.
\begin{exampleblock}{Пример}
\texttt{``face''} е подниза на \texttt{``Please faceAbook''}
\end{exampleblock}  
\end{frame}

\begin{frame}[fragile]{Задачa 4}{Решение}
\lstinputlisting{src/av9/p4.c}
\end{frame}

\begin{frame}{Задачa 5}
Да се напише функција која ќе провери дали дадена текстуална низа е палиндром.
Текстуална низа е палиндром ако се чита исто од лево на десно и од десно на
лево.
\begin{exampleblock}{Пример зборови палиндроми}
dovod\\
ana\\
kalabalak
\end{exampleblock}
\begin{scriptsize}
\emph{За дома:} Да се напише функција која ќе проверува дали одредена реченица е
палиндром. Да се игнорираат празните места, интерпункциските знаци и мали и
големи бувки при споредбата.
\end{scriptsize}
\begin{exampleblock}{Пример реченици палиндроми}
\begin{scriptsize}
Јadenje i pienje daj\\
A man, a plan, a canal, Panama\\
Never odd or even\\
Rise to vote sir
\end{scriptsize}
\end{exampleblock}

\end{frame}

\begin{frame}[fragile]{Задачa 5}{Решение}
\lstinputlisting{src/av9/p5.c}
\end{frame}

\begin{frame}{Задачa 6}
Да се напише функција која за дадена текстуална низа ќе одреди дали е  доволно
сложена да биде лозинка. Секоја лозинка мора да има барем една буква, барем еден
број и барем еден специјален знак.
\begin{exampleblock}{Пример}
\texttt{zdr@v0!} e валидна лозинка.\\
\texttt{zdravo} не е валидна лозинка.
\end{exampleblock}
\end{frame}

\begin{frame}[fragile]{Задачa 6}{Решение}
\lstinputlisting{src/av9/p6.c}
\end{frame}

\begin{frame}{Задачa 7}
Да се напише функција која во стринг што и се предава како влезен параметар ќе
ги промени малите букви во големи и обратно и ќе ги отфрли сите цифри и
специјални знаци.
    
\begin{exampleblock}{Пример}
Ако се внесе: \texttt{``0v@ePr1m3R''} \\
Треба да се добие: \texttt{``VEpRMr''} 
\end{exampleblock}
\end{frame}

\begin{frame}[fragile]{Задачa 7}{Решение}
\lstinputlisting{src/av9/p7.c}
\end{frame}

\begin{frame}[fragile]{Задачa 8}
Да се напише функција која за дадена текстуална низа ќе ги исфрли празните места
на почетокот и крајот на низата.
\begin{exampleblock}{Пример}
Ако се внесе: \begin{verbatim}``   make trim   ''\end{verbatim} \\
Треба да се добие: \texttt{``make trim''} 
\end{exampleblock}
\end{frame}

\begin{frame}[fragile]{Задачa 8}{Решение}
\lstinputlisting{src/av9/p8.c}
\end{frame}
